%\subsection{Complete Markets}
%options and the most common exotic derivatives.  %\bb{(We may impose this structure later but this covers already many payoffs, including exotic ones as shown in the examples below)} 
  With little assumptions on $\alpha,\beta$, we can derive a useful geometric property of the stopping region, termed \textit{ray-connectedness} by  \citet{BroadieDetemple}. %
We first need a scaling property on the  process  $Z_t = (S_t,Y_t) \in \calZ$, $t\in[0,T]$,  with $\calZ$ defined in $\eqref{eq:Z}$. 

% \begin{asm} \label{asm:scale}
%     Given $0\le t \le u \le T$, the stochastic flow $z \mapsto Z_u^{t,z}$  scales  linearly, i.e. %in the sense that 
%     $Z_u^{t,\gamma z} = \gamma Z_u^{t,z}$ $\ \forall \ \gamma \ge 0$.
% \end{asm}

\begin{asm} \label{asm:scale}
    Given $0\le t \le u \le T$ and $\calV_t = \nu$, the stochastic flow $z \mapsto Z_u^{t,(z,\nu)}$  scales  linearly, i.e. 
    $Z_u^{t,(\gamma z,\nu)} = \gamma Z_u^{t,(z,\nu)}$ $\ \forall \ \gamma \ge 0$.
\end{asm}

% \begin{asm} \label{asm:scale}
%     Given $0\le t \le u \le T$ and $\nu \in \calV$, the stochastic flow $z \mapsto X_u^{t,(z,\nu)}$  scales  linearly, i.e. 
%     $X_u^{t,(\gamma z,\nu)} = \gamma X_u^{t,(z,\nu)}$ $\ \forall \ \gamma \ge 0$.
% \end{asm}

The next result is an adaptation of  Proposition A.5 in  \cite{BroadieDetemple}. 

\begin{proposition}
\label{prop:star}
Let $\eta =1$ and assume that $\alpha,\beta$ are homogeneous in $z$, i.e. $\pi(\gamma z)=\gamma \pi(z)$  for all $\gamma>0$,  $\pi \in \{\alpha,\beta\}$. Then \cref{asm:star} holds with the homeomorphism
%If  $\alpha$ is also continuous and positive on $\calX \setminus \{\boldsymbol{0}\}$, 
%then the map 
\begin{equation}\label{eq:homeo}
A=(\alpha,\Xi): \calX  \mapsto \calA \times \calE, \quad 
\Xi(x) = \left(\frac{z}{\alpha(z)},\nu\right), \quad x = (z,\nu). 
\end{equation}
%where $\calA \subseteq \R_+$ and $\calE \subseteq $
% : \calX  \mapsto \calA \times \calE
%with $x = (z,\nu)$, $z = (s,\nu)$. 
%defines a homeomorphism 
%with the following property: If $x,x' \in \calX$ such that  $\Xi(x')=\Xi(x)$, then for $t\in [0,T]$,% and $s \in \calS_t$ for some $t \in [0,T]$, 
%we have 
% \begin{equation*}
%   x \in \calS_t \ \textnormal{ and } \ \ \eta(\alpha(x') - \alpha(x)) \ge 0  \; \Longrightarrow  \;
% x' \in \calS_t.
% \end{equation*} 
 Put differently,  $(z,\nu)\in \calS_t$ implies that $(\gamma z,\nu) \in \calS_t$ for all $\gamma > 1$.
%$$x\in \calS_t \Longrightarrow \gamma x \in \calS_t$$ for any $\gamma > 1$
% \begin{enumerate}
%     \item For a call payoff ($\eta =1$) then $x\in \calS_t \Longrightarrow \gamma x \in \calS_t$ for any $\gamma > 1$.
%     \item For a put payoff ($\eta =-1$) then $x\in \calS_t \Longrightarrow \gamma x \in \calS_t$ for any $\gamma \in (0,1)$.
% \end{enumerate}
\end{proposition} 
\begin{proof}
First, the map $A$ given in $\eqref{eq:homeo}$ is clearly a homeomorphism. Indeed, its inverse $A^{-1}(a,\xi) = (a \xi_z,\xi_\nu)$, $\xi = (\xi_z,\xi_\nu) \in \calE$ is well-defined and continuous. Now let  $x = (z,\nu)$, $x' =(z',\nu')$  such that $\Xi(x')=\Xi(x)$ and $\alpha(z') \le \alpha(z)$. Consequently,   $z'= \gamma z$ with $\gamma := \frac{\alpha(z')}{\alpha(z)} \ge 1$ and $\nu = \nu'$. 

Next, assume that $x\in \calS_t$ for some $t\in [0,T]$. For any  stopping time $\tau \in \calT_{\!t}$, observe for $\pi \in \{\alpha,\beta\}$ that 
$$\pi(X^{t,x'}_{\tau})=\pi(Z^{t,(z',\nu)}_{\tau})=\pi(\gamma Z^{t,(z,\nu)}_{\tau})=\gamma \pi(X^{t,x}_{\tau}), $$ 
using Assumption \ref{asm:scale} and the homogeneity of $\pi$.
Therefore, we obtain
\begin{align*}
    \E^{\Q}\left[ D_{0,\tau} \varphi(X^{t,x'}_{\tau})\right] &= \E^{\Q}\left[ D_{0,\tau} \left(\alpha(X^{t,x'}_{\tau}) -\beta(X^{t,x'}_{\tau}) -K\right)^+ \right]\\
    &= \E^{\Q}\left[ D_{0,\tau} \left(\gamma \alpha(X^{t,x}_{\tau}) -\gamma \beta(X^{t,x}_{\tau}) -K\right)^+ \right]\\
   &\le \gamma \E^{\Q}\left[ D_{0,\tau} \varphi(X^{t,x}_{\tau}) \right] + (\gamma  - 1)K\\
   &\le \gamma \ \varphi(x) + (\gamma  - 1)K\\
   &= \varphi(x').
\end{align*}
In the first inequality, we used  $D_{t,u} \le 1$ and the subadditivity of $x \to x^+$. In the last equality, we noticed that the positive part  in $\varphi$ can be dropped as $x$ belongs to the  stopping region. %Hence $x'\in \calS_t$ and 
%The proof is complete.

\end{proof}

\begin{remark}
For put payoffs, i.e. $\eta =-1$,  \cref{prop:star} applies with the same homeomorphism but different conclusion, namely  $(z,\nu)\in \calS_t \Longrightarrow (\gamma z,\nu) \in \calS_t$ for all $\gamma \in (0,1)$. Moreover, the threshold function $\eqref{eq:thres}$ is given instead by 
$f(t,\xi) = \sup \calA_{t,\xi}$ and is  non-decreasing in $t$  thanks to the time  monotonicity of $\calS$ with respect to  $\subseteq$ (\cref{prop:monot}). 
\end{remark}


\begin{remark}
We recall that the optimal stopping decision is also achieved by comparing %the \textit{intrinsic value}, namely
$\varphi(x)$, oftentimes called the \textit{intrinsic value}, to the \textit{continuation value}
\begin{equation} \label{eq:contValue}
c(t,x) := \underset{\tau \, \in\,  \vartheta(\calT_{t+})}{\text{sup}} \
\E^{\Q}\! \left[ D_{t,\tau}\ \varphi(X^{t,x}_\tau)\ \right],\quad  \calT_{t+} = \calT_t \setminus\{t\}, \quad (t,x)\in [0,T]\times \calX.
\end{equation}
Using $\eqref{eq:payoff}$ and the fact that $\varphi(x)>0$ when $x\in \calS_t$, $t<T$, we obtain for call options that  
$$x \in \calS_t \ \Longleftrightarrow \ \varphi(x) \ge c(t,x) \ \Longleftrightarrow \ \alpha(x) \ge K + \beta(x) + c(t,x).$$
Thus, $f(t,\Xi(x)) = K+ \beta(x)  +  c(t,x)$. The knowledge of the threshold function is therefore equivalent to the knowledge of the continuation value.  Parametrizing  the latter
 has given rise to efficient pricing algorithms as demonstrated by \citet{LSMC} using least square regression and 
 \citet{Kohler} using neural networks.
%Among other methods, the continuation value is at the core of the \citet{LSMC} algorithm, where it is estimated using least square regression. %how our approach differs. 
The key advantage of our  approach  is the guarantee of producing  rational decisions$-$in the sense of Proposition \ref{prop:star}$-$as we exploit the geometric structure of the problem. %thanks the exploitation of the geometric structure of the problem which   %for any given parameter vector $\theta$.  
\end{remark}

%This rather obscure condition will be illustrated with several examples.
%We now 
%discuss several examples. 
%\subsection{Examples}\label{sec:examples}

\subsubsection*{Path-independent Payoffs}
Let us now present important families of path-independent payoffs. Unless stated otherwise,  we assume that no exogenous factors are needed  and write $X=S$. %and Assumption \ref{}. 

\begin{example}\label{ex:vanilla}

Consider vanilla  options of American type.  %example discussed in \cref{sec:1DExample} in an arbitrary model.
Then \cref{asm:star} is fulfilled with the homeomorphism $A = (\alpha,\Xi)$ given by $\alpha(s) = s$, $\beta \equiv 0$ and $\Xi(s) \equiv 1$. %Indeed, we have seen in \cref{sec:1DExample} that $\calS_t$ is a right-unbounded interval, hence $s\in \calS_t$ and  $s' = \alpha(s') \ge \alpha(s) =s$ implies that $s'\in \calS_t$ as well. 
As $\Xi$ is constant, the threshold function $f$ depends solely on time, as we shall see in  \cref{sec:putHeston}.   In a factor model, i.e. $x = (s,\nu)$, we have instead $\alpha(x)=s$, and $\Xi(x) \equiv (1,\nu)$. For a put option, this implies that $(s,\nu) \in \calS_t$ if and only if $s\le f(t,\nu)$; see \cref{sec:putHeston}.   

% and the stopping time in $\eqref{eq:stopTime}$ simply reads  $\tau^{\theta} = \inf\{t \in [0,T] \ | \ S_t \ge G(t; \theta) \} \wedge T.$ 
%As $\Xi$ is constant, the threshold function $G(\cdot;\theta)$ depends solely on time for fixed $\theta$. Therefore, the stopping time in $\eqref{eq:stopTime}$ simply reads $$\tau^{\theta} = \inf\{t \in [0,T] \ | \ S_t \ge G(t; \theta) \} \wedge T.$$
\end{example}

%For the following high-dimensional examples, we assume for simplicity that no exogenous factors are needed (i.e. $l=0$) and write $X=S$. 
%For simplicity, we assume that no exogenous factors are needed (i.e. $l=0$) and write $X=S$. %and Assumption \ref{}. 

\begin{example}\label{ex:bskt}
Consider the class of \textit{basket} options,
$$
\varphi(s)= \left( \eta \left( w^{\top} s - K \right) \right)^{+},
$$
with  weight vector $w$  belonging to the simplex $\Delta_{d} := \{w \in [0,1]^d \ | \ w^\top \mathds{1}=1\} $.
A generalization is given by \textit{index options} where the weights are allowed to vary over time. An example would be a vanilla option written on the S\&P $500$. % where the loadings are proportional to the market capitalization of the underlying stocks. 
In light of $\eqref{eq:payoff}$, we thus choose $\alpha(s) =  w^{\top} s$ and $\beta \equiv 0$. %Furthermore, as $w^\top \Xi(s) = \frac{w^\top s}{\alpha(s)}=1$ for all $s\in \R_+^d$,  
Notice that the image of  $\Xi$ is given by 
$\calE = \{ \xi \in \R_+^d \ | \ w^{\top} \xi =1 \}.$ 
When the weights are uniform, i.e. $\alpha(s) = \frac{1}{d}\sum_{i=1}^d s_i$, the payoff is symmetric in the underlying assets and satisfy interesting properties as outlined in . 

\end{example}

\begin{example}\label{ex:max}
\textit{Max-options} are call options  on the best-performing stock, that is
$$
\varphi(s)= \left(  \max_{i=1,...,d} s_i - K  \right)^{+}.
$$
This gives $\alpha(s)=\max_{i=1,...,d} s_i$ and $\beta\equiv 0$.  
%The subspace $\calE = \{\frac{s}{\alpha(s)} \ | \ s\in \R_+^d\}$ consists of the faces of the $d-$dimensional hypercube not containing the origin. %with at least one coordinate equal to $1$. 
%Hence, 
Moreover, observe that 
$\calE$  is formed by the faces of the  hypercube $[0,1]^d $ not containing the origin. %= \{\xi \in  [0,1]^d \ | \ \max_{i=1,...,d}\xi_i =1 \}

This seemingly innocent family of payoffs reveals many counterintuitive results, as beautifully articulated by \citet{BroadieDetemple}. Among other things, the authors showed that it is never optimal to prematurely exercise a max-call option when more than one asset attains the maximum. In the case $d=2$, this means that the diagonal $\{s \in \R^2_+ \ | \  s_1 = s_2\}$ is entirely contained in the continuation region prior to maturity. As we shall see in \cref{sec:maxCallSym},  the stopping region consists of one connected
component on each side of the diagonal. 

%As the payoff is symmetric, we therefore expect that $\calS_t$, if non-empty,  must be  disconnected for all $t<T$. 
Max-options have become a standard for numerical methods of high-dimensional options; see \cite{Becker1,Becker2} and references therein. 
%We will investigate this contract in further depth in the subsequent sections. 
In contrasts, \textit{put} options on the maximum of $d$ assets are less popular in the literature due to their low premium and are not discussed here. 
Also, one can imagine an option on the \textit{minimum} of several underlyings; see \cite[Chapter 6]{DetempleBook} and \cite{DetempleMin} for a thorough treatment in the two-dimensional case. %of min-call options on two assets.
% illustrate this fact and investigate other propthe 
\end{example}

\begin{example}\label{ex:spread}
Consider the family of \textit{spread options} \cite{Carmona},  that is 
$$
\varphi(s)= \left( \eta  \left ( s_1 - \sum_{i=2}^{d}\gamma_i s_i - K \right) \right)^{+},
$$
where  $(\gamma_{i})_{i=2}^d \in \R^{d-1}_+$ are conversion factors. We naturally set $\alpha(s)=s_1$, $\beta(s)=\sum_{i=2}^{d}\gamma_i s_i$. The corresponding homeomorphism is therefore given by 
$A=(\alpha,\Xi)$ with  $\Xi(s)=\frac{s}{s_1}.$ The threshold function thus depends on all stock prices expressed in terms of $s_1$. 
Choosing an asset as numéraire %(here $s_1$)  
was originated by \citet{Margrabe} for exchange options (i.e. the case $d=2$  and $\gamma_2=1$). 
\end{example}

% Let us verify
% that Assumption \ref{asm: homeomorph} holds with $\alpha$ defined above. As in the proof of Proposition \ref{prop:star}, take $s\in \calS_t$ for some $t\in [0,T]$ and $s'\in \R^d_+$ such that  $\frac{s}{s_1}=\frac{s'}{s'_1}$ and $s_1' \ge s_1$. For any stopping time $\tau \in \calT_{t}$, we obtain similarly
% \begin{align*}
%     \E^{\Q}\left[ D_{0,\tau} \varphi(S^{t,s'}_{\tau})\right] &= \E^{\Q}\left[ D_{0,\tau} \left(\psi_1\,  S^{t,s'}_{\tau,1} - \sum_{i=2}^{d}\psi_i S^{t,s'}_{\tau,i} -K\right)^+ \right]\\
%      &= s'_1\E^{\Q}\left[ D_{0,\tau} \left(\psi_1\,  S^{t,\Xi(s')}_{\tau,1} - \sum_{i=2}^{d}\psi_i S^{t,\Xi(s')}_{\tau,i} -\frac{K}{s_1'}\right)^+ \right]\\
%     &\le \E^{\Q}\left[ D_{0,\tau} \left(\gamma \pi(X^{t,x}_{\tau}) -K\right)^+ \right]\\
%   &\le \gamma \E^{\Q}\left[ D_{0,\tau} \left( \pi(X^{t,x}_{\tau}) -K\right)^+ \right] + (\gamma  - 1)K\\
%   &\le \gamma \left( \pi(x) -K\right) + (\gamma  - 1)K\\
%   &= \varphi(x').
% \end{align*}

% \begin{align*}
%     \E_t\left[ \varphi(t,S_{\tau}^{t,s})\right] &= s_1 \E_t\left[e^{-r\tau} \left(\frac{K}{s_1}+   \frac{\psi_2}{d-1}\sum_{i=2}^{d} S^{i,t,\Xi(s)}_{\tau} - \psi_1 S^{1,t,\Xi(s)}_{\tau} \right)^+ \right]\\
%     &\leq  s_1 \E_t\left[e^{-r\tau} \left(\frac{K}{s'_1}+   \frac{\psi_2}{d-1}\sum_{i=2}^{d} S^{i,t,\Xi(s')}_{\tau} - \psi_1 S^{1,t,\Xi(s')}_{\tau} \right)^+ \right] + K e^{-rt}\left(1- \frac{s_1}{s_1'}\right)^+\\
%     &=  \frac{s_1}{s_1'} \E_t\left[e^{-r\tau} \left(K+   \frac{\psi_2}{d-1}\sum_{i=2}^{d} S^{i,t,s'}_{\tau} - \psi_1 S^{1,t,s'}_{\tau} \right)^+ \right]+ K e^{-rt}\left(1- \frac{s_1}{s_1'}\right) \quad (s_1\leq s_1' )\\
%     &\leq \frac{s_1}{s_1'} \varphi(t,s')+ K e^{-rt} \left(1- \frac{s_1}{s_1'}\right)\\
%     &= e^{-rt}\left(K + \frac{\psi_2}{d-1}\sum_{i=2}^{d} \frac{s'_{i}}{s_1'}s_1 - \psi_1 s_1 \right)\\
%     &= e^{-rt}\left(K + \frac{\psi_2}{d-1}\sum_{i=2}^{d} s_{i} - \psi_1 s_1 \right) \quad (\text{since } \Xi(s)=\Xi(s'))\\
%     &= \varphi(t,s),
% \end{align*}
% %This is inspired from the Margrabe's technique for exchange options \cite{Margrabe} to set a stock as numéraire. % and express all assets  in terms of the latter. 
 

% \end{example}

%\subsubsection*{Path-dependent Payoffs}

\subsubsection*{Path-dependent Claims}

%We finish this 
%section with path-dependent claims. 
We assume throughout that $d=1$, although multi-asset path-dependent options may exist (see  Example \ref{ex:Asian}). % (e.g. an American Asian call on an equity index). 
%We now have to specify path-dependent variable

\begin{example}\label{ex:Asian}
\textit{Asian options} %(see \citet{Hull}, Section 25.12) 
are %vanilla 
derivatives with payoff involving the running average of the  underlying asset(s).  With this in mind, it is therefore natural to set $Y_t=\Upsilon(\SSS_t) = \frac{1}{t} \int_0^t S_u du$. It is easily seen that $X=(S,Y)$ is a Markov process.
There are two types of Asian payoffs, namely (i) Fixed strike: $\varphi(x) = \left(\eta (y - K)\right)^{+}$ and (ii) Floating strike:  $\varphi(x) = \left(\eta (s - \gamma y)\right)^{+}$ ($\gamma$: scaling factor).
% \begin{enumerate}
%     \item Fixed strike: $\varphi(x) = \left(\eta (y - K)\right)^{+}$.
%     \item Floating strike:  $\varphi(x) = \left(\eta (s - \gamma y)\right)^{+}$, where $\gamma$ is a scaling factor.
% \end{enumerate}

The projection functions are therefore $\alpha(x)=y$, $\beta \equiv 0$ and  $\alpha(x)=s$, $\beta(x) = \gamma y$ for fixed and floating strike options, respectively. %Although not covered in the numerical experiments, 
A multi-dimensional extension  could be obtained for instance by replacing the underlying by a stock index, e.g.
$$ \Phi(\SSS_t) = \left( \frac{1}{t} \int_0^t w_u^{\top} S_u du - K\right)^{+}, \quad S_u\in \R^d, \quad w_u\in \Delta_{d},\quad d\ge 2.$$

%Proposition \ref{prop:star} clearly applies, giving the homeomorphism
%$A=(\alpha,\Xi)$, with $\alpha = \pi$ and $\Xi(x) = (\frac{s}{y},1)$. 
%The threshold function will therefore depend on the ratio $\frac{S_t}{Y_t}$, which is used characterized the trend of the stock in path-dependent volatility models (cite Guyon)

%The first coordinate of the process $\Xi(X_t)=\frac{x}{y}$ is the ratio $\frac{S_t}{Y_t}$, 
% which is used characterized the trend of the stock in path-dependent volatility models (cite Guyon)?
\end{example}


\begin{example}\label{ex:lookback}
\textit{Lookback options} %(see \citet{Hull}, Section 25.10) %allows its holder to get
provides exposure to the minimum or maximum values attained by the stock so far. This suggests to choose $$Y_t=\Upsilon(\SSS_t) = \left(\max_{u\in[0,t]} S_u,\min_{u\in[0,t]} S_u\right).$$
It is easily seen that $X=(S,Y)$ is a  Markov process with state space $\calX = \{x=(s,y_1,y_2) \in \R_+^3 \ | \ y_2 \le s \le y_1\}$. Both fixed and floating contracts exist as in Example \ref{ex:Asian}. The most common contracts with their corresponding projection functions are listed in Table \ref{tab:lkbk}. The scaling factor $\gamma$ appearing in the payoff of floating strike options is introduced to reduce the price of these otherwise expensive contracts. We therefore choose $\gamma \in [1,\infty)$ and $\gamma \in (0,1]$ for call and put options, respectively.  
When $\gamma=1$, the payoff of a floating strike lookback call (put) option is precisely the %time $t$ 
drawdown (drawup) of the stock. See \citet{DaiKwok} for a thorough treatment of American lookback claims.  

%Note that the stopping decision relies on the size of the relative drawdown or drawup. 

\begin{table}[H]
    \centering
    \begin{tabular}{r|c|c|c}
        Option & $\Phi(\SSS_t)$ & $\alpha(x)$ & $\beta(x)$ \\ \hline \hline
         Fixed Strike Call & $(\max_{u\in[0,t]} S_u - K)^{+} $ & $y_1$ & $0$ \\[0.2em] \hline
       Fixed Strike Put & $(K - \min_{u\in[0,t]}S_u )^{+} $ & $y_2$ & $0$ \\[0.2em] \hline
        Floating Strike Call & $S_t - \gamma\min_{u\in[0,t]} S_u$  & $s$ & $\gamma  y_2$ \\[0.2em] \hline
        Floating Strike Put & $\gamma\max_{u\in[0,t]} S_u - S_t$ & $\gamma y_1$ & $s$ \\[0.2em] \hline \hline
    \end{tabular}
    \caption{Lookback Options, $x=(s,y_1,y_2)$.}
    \label{tab:lkbk}
\end{table}

%In addition, the projection functions are the same.
\end{example}


% \begin{remark}
% Floating strike options. The general form is
% $$
% \Phi(\SSS_t):= \left( \eta  \left [ S_t - k\Psi(\SSS_t) \ \right] \right)^{+},
% $$
% where $\Psi$ is typically the running maximum (floating lookback) or the average (floating Asian) and $k$ is a scaling factor. 
% We therefore set $Y_t=\Upsilon(\SSS_t) = \Psi(\SSS_t)$, $\alpha(x) = s$ and $\beta(y)=ky$. 

% \end{remark}

% \begin{example}
% \bb{\textit{Barrier options}}. Given in Table \ref{tab:barrier}. We can  set  $Y_t=\Upsilon(\SSS_t) = (\max_{u\in[0,t]} S_u,\min_{u\in[0,t]} S_u)$, $\beta \equiv 0$ and $\alpha$ as in the right column of Table \ref{tab:barrier}. As $\alpha$ is not homogeneous (and often $0$ for calls) we cannot apply Proposition \ref{prop:star}. Moreover, the stopping region is ray-connected for some payoffs (e.g. Up & and In Call) but not all (e.g. Up \& Out Call). \bb{Is it worth mentioning this class? }
% \end{example}

% \begin{table}[H]
%     \centering
%     \begin{tabular}{r|c|c}
%         Option & $\Phi(\SSS_t)$ & $\alpha(s,y_1,y_2)$  \\ \hline \hline
%          Up \& In Call & $(S_t - K)^{+} \mathds{1}_{\left\{\max_{u\in[0,t]} S_u \ge B\right\}}$ & $s \mathds{1}_{\{y_1 \ge B\}}$ \\ \hline
%         Up \& Out Call & $(S_t - K)^{+} \mathds{1}_{\left\{\max_{u\in[0,t]} S_u \le B\right\}}$ & $s \mathds{1}_{\{y_1 \le B\}}$ \\ \hline
%       Down \& In Call & $(S_t - K)^{+} \mathds{1}_{\left\{\min_{u\in[0,t]} S_u \le B\right\}}$ & $s \mathds{1}_{\{y_2 \le B\}}$ \\ \hline
%         Down \& Out Call & $(S_t - K)^{+} \mathds{1}_{\left\{\min_{u\in[0,t]} S_u \ge B\right\}}$ & $s \mathds{1}_{\{y_2 \ge B\}}$ \\ \hline \hline
%         Up \& In Put & $( K-S_t)^{+} \mathds{1}_{\left\{\max_{u\in[0,t]} S_u \ge B\right\}}$ & $\max(s, K \mathds{1}_{\{y_1 < B\}})$ \\ \hline
%         Up \& Out Put & $( K-S_t)^{+} \mathds{1}_{\left\{\max_{u\in[0,t]} S_u \le B\right\}}$ & $\max(s, K \mathds{1}_{\{y_1 > B\}})$ \\ \hline 
%       Down \& In Put & $( K-S_t)^{+} \mathds{1}_{\left\{\min_{u\in[0,t]} S_u \le B\right\}}$ & $\max(s, K \mathds{1}_{\{y_2 > B\}})$ \\ \hline
%         Down \& Out Put & $( K-S_t)^{+} \mathds{1}_{\left\{\min_{u\in[0,t]} S_u \ge B\right\}}$ & $\max(s, K \mathds{1}_{\{y_2 < B\}})$ \\ \hline
%     \end{tabular}
%     \caption{Barrier Options}
%     \label{tab:barrier}
% \end{table}
