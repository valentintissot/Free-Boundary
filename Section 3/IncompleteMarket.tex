
% \subsection{Incomplete Markets}
% We now allow the presence of  non-tradable factors  $\calV=(\calV_{\cdot,1},\ldots,\calV_{\cdot,m})$. which makes the market incomplete. 
% Obviously, characterizing the stopping region is hopeless for a general process $\calV$. We therefore need to narrow down the scope and specify the nature of each factor. 
% %Note that Assumption \ref{asm: scale2} might not hold for a general process $\calV$ and is therefore too restrictive.   

% We here focus on stochastic volatility models of the form 
% \begin{equation}\label{eq:SVM}
%     \frac{dS_{t,i}}{S_{t,i}} = (r-\delta_i)dt + \sqrt{\calV_{t,i}} \, dW_{t,i}, \quad S_{0,i} \in \R_+, \quad i=1,\ldots,d, 
% \end{equation}
% where $W$ is Brownian motion in $\R^d$. In other words, we have $m=d$ and $\calV_{i,1}$ is the stochastic variance of the stock $S_{t,i}$. When each $\calV_{i,1}$ is a CIR process, that is 
% $$ d\calV_{t,i} = (\kappa_i(\bar{\calV}_i-\calV_{t,i}) - \gamma_i^{\Q}\calV_{t,i})dt + \sigma_i \sqrt{\calV_{t,i}}\  d\tilde{W}_{t,i},$$
%  then this boils down to a multi-dimensional Heston model. The parameters $\kappa_i$, $\bar{\calV}_i$, $\gamma_i^{\Q}$ are respectively the  mean reversion speed, long-term average and the market price of volatility risk. 

% When $d=1$ and $\varphi$ is a bounded, convex function,  \citet{LambertonHeston} showed that the value function $v(t,x)$, $x=(s,\nu)$, is  non-decreasing in $\nu$ for all $t\in [0,T]$. %\bb{(+ Touzi's paper for SDEs with Lipschitz coefficients)}.
% Consequently, the  segment $\{s\} \times [0,\nu]$ %\{(s,\gamma \nu) \ | \ \gamma \in [0,1]\}$ 
% is contained in $\calS_t$ whenever $(s,\nu) \in \calS_t$.

% This suggests to write 
% \begin{equation}\label{eq:ThresholdS}
% \calS_t = \{x=(s,\nu)\in \calX \ | \ \nu \le f(t,s)\},
% \end{equation}
% for a threshold function $f:[0,T] \times \R_+ \to \R_+$. 
% Conversely, for call and put options, we can also take the homeomorphism  $A(x) =(\alpha(x),\Xi(x)) =  (s,\nu)$ ($=x$), which gives 
% \begin{equation}\label{eq:ThresholdNu}
%     \calS_t = \{x=(s,\nu)\in \calX \ | \ \eta (s - f(t,\nu))\ge 0 \},
% \end{equation}
% with $f:[0,T] \times \R_+ \to \R_+$ such that $f(t,\nu) \to K$ as $\nu \downarrow 0$ or $t\uparrow T$. \bb{The formulation $\eqref{eq:ThresholdS}$ is the same for puts and calls, while $\eqref{eq:ThresholdNu}$ may be more intuitive. We can see which one works better numerically!}
% \begin{remark}
% As it is never optimal to exercise  a max call option when two stocks attains the maximum, we expect that 
% $$\lim_{x_2 \to 1} f(t,x) = \infty, \quad x_1 = \alpha(s). $$
% Therefore $f(t,[0,1]) = [f_1(t),\infty]$ where $f_1$ is the optimal one-dimensional boundary.
% \end{remark}
% ... 

% When $\varphi$ is also non-increasing (e.g. a put payoff), this suggests to consider $\alpha(x)=s$, $\Xi(x) = \nu$, giving the representation 
% \begin{equation}\label{eq:ThresholdNu}
%     \calS_t = \{x=(s,\nu)\in \calX \ | \ s \le f(t,\nu)) \},
% \end{equation}
% with $f:[0,T] \times \R_+ \to \R_+$ such that $f(t,\nu) \to K$ as $\nu \downarrow 0$ or $t\uparrow T$. This implies that the rectangle $[0,s] \times [0,\nu]$ is contained in $\calS_t$ as soon as $(s,\nu)$. 




