%\subsection{Functional It\^o Calculus}
We fix throughout a horizon $T>0$, which can be interpreted as the maturity of a financial derivative. Let $\Lambda_t = \calD([0,t],\R)$ be the Skorokhod space of c\`adl\`ag paths of length $t\in [0,T]$. 
Given $X_t\in \Lambda_t$, $X_s$ denotes the whole trajectory up to time $s\le t$, while $x_s= X_t(s)$ is the value at time $s$. 
Moreover, write $\Lambda := \bigcup_{t\in[0,T]}\Lambda_t$ for the collection of all c\`adl\`ag paths. The distance between elements of $\Lambda$ is measured according to %can  and define the metric
$$d_{\Lambda}( X_t, Y_s) =t-s + \sup_{0 \le u \le t} |x_u - y_{u\wedge s} | , \quad 0 \le s \le t \le T. $$ %\lVert X_t - Y_{s,t-s} \rVert_{\infty}
%the flat extension $X_{t,\delta t}(s) = x_{s \wedge t}$, $s\in [0,t+\delta t ]$. 
See \cite{Dupire} for other topological considerations. % on the topology of $\Lambda$.  
We adorn $\Lambda$  %with a $\sigma-$algebra $\calF$, filtration $\F$ and probability measure $\Q$ %(e.g., the Wiener measure) 
%to form 
with a stochastic basis
$(\Lambda,\calF,\F, \Q)$. %With a slight abuse of notation, w
We interpret $X \in \Lambda$ both as a path and the coordinate process. To disambiguate the notations, we will often use the letter $X$ when referring to a  specific path and $Y$ when taking (conditional) expectation under $\Q$. %for the (random) coordinate process% where $\calF$ is the natural filtration of the coordinate process $X_t(\omega)=\omega_t$

A \textit{functional} transforms a path into a scalar, i.e. it is a map $f:\Lambda \to \R$. We make the distinction between functionals defined on $\Lambda$ and those restricted to trajectories of maximal length, i.e. $g:\Lambda_T \to \R$. The latter are called \textit{$T-$functionals}.  
%Lastly, we introduce important classes of functionals. 
For $p\in [1,\infty]$ and $t\in [0,T]$, 
let $L^p(\Lambda_t)$ be the subspace of functionals $f$ restricted to $\Lambda_t$ such that  
\begin{equation}\label{eq:L(Lambda_t)}
    \lVert f \rVert_{L^p(\Lambda_t)} :=  \lVert f(Y_t) \rVert_{L^p(\Q)} < \infty.
\end{equation}
Furthermore, we define 
\begin{equation}\label{eq:L(Lambda)}
    L^p(\Lambda) = \left \{ f: \Lambda\to \R \ \Big | \ \lVert f \rVert_{L^p(\Lambda)} := \sup_{t \in [0,T]} \lVert f \rVert_{L^p(\Lambda_t)} < \infty \right\}
\end{equation}
 Clearly, $f\in L^p(\Lambda) $ implies that $f|_{\Lambda_t} \in L^p(\Lambda_t)$ for all $ t \in [0,T]$. The converse is  false in general, as shown by the deterministic functional $f|_{\Lambda_t}  \equiv \frac{1}{t}\mathds{1}_{\{t>0\}}$. The interested reader may verify that $\lVert \cdot \rVert_{L^p(\Lambda)}$ is a norm and $L^p(\Lambda)$ a Banach space. % \bb{(?)}.

%In essence, the functional Taylor expansion separates the idiosyncratic