%\section*{Introduction}


The stochastic Taylor expansion  \cite{KP} %[Chapter~5] 
 lies at the core of numerical 
 integration of 
 stochastic differential equations.
It provides explicit weak/strong 
error bounds for discretization
schemes, which is of great relevance for the valuation of European derivatives. %, e.g. Euler-Maruyama, Milstein-Platen, to name a few.
 %It permits the construction of numerical schemes where the approximation error (weak or strong) can be made explicit. 
 %It seems therefore natural to extend this tool to the framework of path-dependent functionals. 
% For instance, the Monte Carlo pricing error with trajectories from the Euler schemes  is controlled by the step size of the discretization in a linear fashion.
% In the valuation of European options, the Euler-Maruyama scheme yields a pricing (or weak) error proportional to the step size of the discretization. For path-dependent payoffs, one is rather interested in the hedging (or strong) error of a numerical method.
%For instance, the Euler-Maruyama method (\cite{Maruyama}) yields a pricing/weak error for power options that is proportional to the discretization step size.
For instance, the Euler-Maruyama method \cite{Maruyama} yields a weak error for the expectation of monomials %—or in financial terms, the price of European power options—
that is proportional to the time step.  %discretization step size. %In financial terms, this means that power options are priced 
By extension, it seems natural to seek %look for 
test functionals akin to monomials which characterize weak convergence for path-dependent payoffs. 

A good candidate is the family of \textit{signature functionals}. The path signature has gained much attention in recent years, due in particular %to its  ability to compress a path efficiently and 
to its universal approximation property \cite{LyonsNum,Szpruch}. In fact, a density result à la Stone-Weierstrass exists \cite{Hao}. %for these functionals,
In light of this, the signature functionals
plays therefore the role of monomials in the path space. %therefore 
This parallel is further reinforced %also
%apparent
in cubature methods \cite{LyonsVictoir,Crisan}   generalizing Gaussian quadrature rules to the Wiener space; %orthogonal polynomials %whose integral must match under a discrete measure 
%are replaced by signature elements.
the exact integration of orthogonal polynomials 
turns into a perfect fit of expectated signature elements. \bb{+ \ldots}
%under a discrete measure 
% boils down to matching expectation of signature elements. turns into a perfect fit of expectated signature elements.

%becomes a moment matching condition for signature elements. %whose integral must match under a discrete measure 
%are replaced by signature elements.

%The matching of integrals 
%  matching the expectation of signature elements under the Wiener measure. 
 
%In this paper
In response to this appeal, % these challenges, 
we shed further light on the stochastic Taylor expansion for  path-dependent payoffs, which we call the \textit{functional Taylor expansion}. %Without much surprise, 
This is made possible by 
tools from the functional Itô calculus \cite{Dupire}. To the best of our knowledge, this generalization has been first proposed by \cite{LittererOberhauser} %where the underyling path is 
in the diffusion case.    
In this is work, we intend to bring  clarity to this deep result and establish a parallel with the Wiener-chaos expansion \cite{DiNunno,Nualart}. 
%\bb{+\ldots}


%Fréchet (Hille-Phillips) vs Functional Itô expansion? 

%We believe that this connection will help the reader 
%to further gain intuition into this beautiful yet technical result.
% in a natural framework.
%Due to their density property, the signature functionals play the role of the monomials in the space of functionals.

%The remainder of this work is organized as follows. \bb{...}

 
   
 
%  \bb{Cubature:} Lyons-Victoir, Crisan, Filipovic et al. (towards American options).