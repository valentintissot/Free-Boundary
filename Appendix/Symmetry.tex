\subsubsection{Symmetry}\label{sec:symmetry}

Many multi-asset, path-independent derivatives  present a type of symmetry in their payoff; see  \cref{ex:bskt} and \ref{ex:max}. Under certain conditions, this will carry over to the  stopping region. 
We assume $X=S$ throughout and  first recall the definition of symmetric functions. 

\begin{definition}
  A  function $\varphi: \R_+^d \to \R$ is called  \textit{symmetric} if it is invariant under permutation of its argument. Put differently, it must satisfy
    $$\varphi \circ \varsigma = \varphi, \quad \forall \ \varsigma \in \calP_{\! d} :=  \{\varsigma: \R^d_+ \mapsto \R^d_+ \, |\, \{s'_1,\ldots,s'_d\}=\{s_1,\ldots,s_d\}, \ s' = \varsigma(s) \}.$$%\{\varsigma: \R^d_+ \mapsto \R^d_+ \, |\, \varsigma(s)=(s_{i_1},...,s_{i_d}), \{i_1,...,i_d\} = \{1,...,d\} \}
   %with the group of permutations  $\calP_{\! d} =  \{\varsigma: \R^d_+ \mapsto \R^d_+ \, |\, \varsigma(s)=(s_{i_1},...,s_{i_d}), \{i_1,...,i_d\} = \{1,...,d\} \}.$
    %where $\calP_{\! d}$ is the group of permutations, 
    %$$\calP_d = \{\varsigma: \R^d_+ \mapsto \R^d_+ \, |\, \varsigma(s)=(s_{i_1},...,s_{i_d}), \{i_1,...,i_d\} = \{1,...,d\} \}.$$
\end{definition}

We also make the following distributional assumption on the stock prices.

\begin{asm}\label{asm: id} For all $(t,s)\in [0,T]\times \R_+^d$, the law of the forward-starting process $S^{t,s}$  is invariant under permutation, that is
    \begin{equation}\label{eq:id}
        \varsigma(S^{t,s}) \circ \Q = S^{t,\varsigma(s)} \circ  \Q, \quad \forall \ \varsigma \in \calP_{\!d}.
    \end{equation}
    % \begin{equation}
    %     \Q \circ (S^{t,s})^{-1} \circ \varsigma^{-1}  = \Q \circ (S^{t,\varsigma(s)})^{-1},
    % \end{equation}

    %The stock prices are identically distributed when starting at the same point, i.e. 
    %$$ \sigma(S^{t,\mathds{1}})  \circ \Pb = S^{t,\mathds{1}}  \circ \Pb \quad \forall \,\sigma \in \calP_d.$$
\end{asm}

Note that when Assumption \ref{asm: id} fails to hold,  the stopping region is in general not symmetric, regardless of the symmetry of $\varphi$; see \cref{sec:maxCallAsym}. 
The following result is now immediate.
\begin{proposition}
\label{prop:sym} Let $\varphi$
 be a symmetric payoff. If Assumption \ref{asm: id} holds, then the $t-$sections of the stopping region associated to $\varphi$ satisfy
$\varsigma(\calS_t)= \calS_t$ whatever $ \varsigma \in \calP_{\!d}$.
\end{proposition}
\begin{proof} Fix $t \in [0,T]$ and an arbitrary permutation $\varsigma \in \calP_d$. As $\varphi$ is symmetric, we show equivalently that $v(t,\cdot) \circ \varsigma =v(t,\cdot)$ with $v$ the associated value function. 
Given $\tau \in \calT_{\!t}$, we have
\begin{align*}\label{eq:proofSym}
    \E^{\Q}[\varphi(\tau,S^{t,s}_\tau)] =  \E^{\Q}[\varphi(\tau,\varsigma(S^{t,s}_\tau))] \
    \overset{\eqref{eq:id}}{=} \ \E^{\Q}[\varphi(\tau,S^{t,\varsigma(s)}_\tau)].
\end{align*}
Taking the supremum over all stopping times %on both sides %of $\eqref{eq:proofSym}$ 
yields the claim.
\end{proof}
Under the hypotheses of Proposition \ref{prop:sym}, it is sufficient to characterize the $t-$sections,  
$$\calV_t= \calS_t \cap \calO, \quad t \in [0,T], $$ where $\calO = \{s\in \R_+^d \,|\, s_1 \geq \ldots \geq s_d\}$ denotes the \textit{cone of ordered stocks}. The stopping region can be recovered by shuffling coordinates, namely $\calS_t = \bigcup_{\varsigma \in \calP_{\!d}} \varsigma(\calV_t)$ for $t\in[0,T]$.\footnote{Indeed, if $s\in  \bigcup_{\varsigma \in \calP_{\!d}} \varsigma(\calV_t)$, then $s=\varsigma(s')$ for some pair $(\varsigma,s') \in \calP_d \times \calV_t$. Hence 
$s \in \calS_t$ thanks to Proposition \ref{prop:sym}. Conversely for $s\in \calS_t$, there exists $\bar{\varsigma} \in \calP_d $ s.t. $s':=\bar{\varsigma}(s) \in \calO$ ($\bar{\varsigma}$ simply sorts  $s$ in non-increasing order). Proposition \ref{prop:sym} implies that $s' \in \calS_t$, which in turn gives   $ s' \in \calV_t$ and $s = \bar{\varsigma}^{\ -1}(s') \in \bigcup_{\varsigma \in \calP_d} \varsigma(\calV_t)$.} 
This fragmentation present benefits from a computational standpoint; see \cref{sec:maxCallSym}. 

 %Indeed, consider a max-call option on $d=2$ assets in the Black-Scholes model with same volatility but different dividend rates. Then \bb{...}