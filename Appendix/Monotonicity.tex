
%This property is valid for path-dedenpendent payoffs as well. 

\begin{proposition}\label{prop:monot}
Suppose that  $X$ and $D$ are stationary, that is $   
        X_u^{t,x} \circ \Q = X_{u-t}^{0,x} \circ  \Q \ $  and $D_{t,u} = D_{0,t-u}$ for every $t \le u $. 
Then the exercise boundary  expands over time, i.e. 
$ \calS_t \subseteq \calS_{t'}$ for all $t \le  t' $.

\end{proposition}
\begin{proof}
Take $t,t'\in [0,T]$ such that $\delta t := t'-t \ge 0$. %\footnote{The shift operator can be used, too.} 
 For  $\tau \in \calT_{t'}$, we have
\begin{align*}
    \E^{\Q}[D_{t',\tau}\varphi(X^{t',x}_{\tau})] &= \E^{\Q}[D_{t,\tau - \delta t }\varphi(X^{t,x}_{\tau - \delta t})] \le v(t,x), 
\end{align*}
as $\tau - \delta t \in \calT_t \cap \ [t,T-\delta t] \subseteq \calT_t$. If  $x\in \calS_t$, this gives   
$\E^{\Q}[D_{t',\tau}\varphi(X^{t',x}_{\tau})] \le  \varphi(x)$ for all $ \tau \in \calT_{t'} $ so  $x \in \calS_{t'}$ as claimed. 
\end{proof}

% \begin{corollary}
% \label{cor:monot} Let $\xi \in \calE$ and $f$  as  in $\eqref{eq:thres}$. Then $t \mapsto f(t,\xi)$  is non-increasing.
% \end{corollary}
% \begin{proof}  Fix $\xi \in \calE$ and suppose that $a \in \calA_{t,\xi}$, i.e. $A^{-1}(a,\xi) \in \calS_t$. If $t' \ge t$, then $A^{-1}(a,\xi) \in \calS_{t'}$ as well from which we conclude that $\calA_{t,\xi} \subseteq \calA_{t',\xi}$. Thus $f(t',\xi) = \inf \calA_{t',\xi} \le \inf \calA_{t,\xi} = f(t,\xi)$. 
% \end{proof}

% \begin{remark}
% If we  assume instead that the homeomorphism satisfies
% $$\left [ s\in \calS_t,\,  \Xi(s)=\Xi(s'), \,  \alpha(s')\geq \alpha(s) \right]  \Longrightarrow s'\in \calS_t,$$ (e.g. for a call-type payoff), then 
% $$f(t,x)= \inf \{a \in \R \, |\, A^{-1}(a,x) \in \calS_t\},$$
% and Corollary \ref{cor:monot} implies  that $t\mapsto f(t,x)$ is non-increasing.
% \end{remark}