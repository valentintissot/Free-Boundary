\subsection{Convexity}
Knowing topological properties of the unknown stopping region can help to characterize it more efficiently. We here focus on convexity. As the stopping region may not be connected (see \cref{ex:max}), we aim at determining when each connected component is convex.  We start off with some assumptions. 

\begin{asm}\label{asm:convlin}
$x \mapsto \varphi(x)$ is convex on $\calX$ and affine within each connected component of  $\calS_t$. 
\end{asm}

% Note that Assumption \ref{asm: convlin} is fulfilled by all payoffs from Section \ref{sec:examples}. Indeed, this is immediate when the functions $\alpha,  \beta$ in $\eqref{eq:payoff}$ are affine and for the max-option, the payoff is the positive part of a non-decreasing convex function, hence convex.   
We also need the following condition on $X$, which holds for instance for arithmetic and geometric Brownian motions. 
\begin{asm} \label{asm:affine}
    Given $0\le t \le u \le T$, the stochastic flow $x \mapsto X_u^{t,x}$  is an affine function.
\end{asm}

If both \cref{asm:scale} and  \ref{asm:affine}  hold, then necessarily $X_u^{t,x} = x \odot X_u^{t,\mathds{1}}$ so  the process is of "geometric" type. This leads us to the following result. 

\begin{proposition}
Under \cref{asm:convlin} and \ref{asm:affine}, $\calS_t$ is convex within each connected component.
\end{proposition}

\begin{proof}
Let $x,x'\in \calS_t$ belonging to the same connected component and  $\tilde{x}=\gamma x + (1-\gamma)x'$ for $\gamma \in [0,1]$. Then for all $\tau \in \calT_t$,
\begin{align*}
\E^{\Q}[D_{t,\tau}\varphi(X^{t,\tilde{x}}_\tau)] &=%\overset{\ref{asm:affine}}{=} 
\E^{\Q}[D_{t,\tau} \varphi(\gamma X^{t,x}_\tau + (1-\gamma) X^{t,x'}_\tau)]\\
    &\leq%\overset{\ref{asm:convlin}}{\leq}
    \gamma \E^{\Q}[D_{t,\tau}\varphi(X^{t,x}_\tau)]
    +
    (1-\gamma)\E^{\Q}[D_{t,\tau}\varphi(X^{t,x'}_\tau)]\\
    &\leq \gamma \varphi(x) + (1-\gamma) \varphi(x')\\
    &= %\overset{\ref{asm:convlin}}{=}
    \varphi(\tilde{x}).
\end{align*}
\end{proof}
Working with convex stopping regions is convenient as a continuum of stopping points can be deduced solely based on a few elements of $\calS_t$. Indeed, if we identify $x^{1},...,x^{J}$ as belonging to  $\calS_t$ %using the neural net $\Phi(\cdot;\theta)$,
, then   $\text{conv}(\{x^{1},...,x^{J}\}) \subseteq \calS_t$ as well.